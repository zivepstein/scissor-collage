	\documentclass[a4paper,UKenglish]{lipics-v2016}
%This is a template for producing LIPIcs articles. 
%See lipics-manual.pdf for further information.
%for A4 paper format use option "a4paper", for US-letter use option "letterpaper"
%for british hyphenation rules use option "UKenglish", for american hyphenation rules use option "USenglish"
% for section-numbered lemmas etc., use "numberwithinsect"
 
\usepackage{microtype}%if unwanted, comment out or use option "draft"
\usepackage{csquotes}

%\graphicspath{{./graphics/}}%helpful if your graphic files are in another directory

\bibliographystyle{plain}% the recommended bibstyle
   %Proceedings number goes here!

% Author macros::begin %%%%%%%%%%%%%%%%%%%%%%%%%%%%%%%%%%%%%%%%%%%%%%%%
\title{Color Preserving Image Transformation with Scissor Collage}
\titlerunning{Visualizing Scissors Congruence} %optional, in case that the title is too long; the running title should fit into the top page column


\author[1]{Ziv Epstein}
\author[1]{Robin Pollak}
\author[1]{Dmitriy Smirnov}
\affil[1]{Computer Science Department, Pomona College\\
  Claremont,CA\\
  \{\texttt{ziv.epstein, robin.pollak, dmitriy.smirnov}\}\texttt{@pomona.edu}}

\authorrunning{ Z.\,G. Epstein, R. Pollak, and D. Smirnov.} %mandatory. First: Use abbreviated first/middle names. Second (only in severe cases): Use first author plus 'et. al.'

\Copyright{Ziv G. Epstein, Robin Pollak \& Dmitry Smirnov}%mandatory, please use full first names. LIPIcs license is "CC-BY";  http://creativecommons.org/licenses/by/3.0/

\subjclass{I.3.5 Computational Geometry and Object Modelling -- Geometric Algorithms, languages and systems, K.3.1 Computer Uses in Education -- Computer-assisted instruction}% mandatory: Please choose ACM 1998 classifications from http://www.acm.org/about/class/ccs98-html . E.g., cite as "F.1.1 Models of Computation". 
\keywords{polygonal congruence, color matching, image transformation, geometry, rigid transformations}% mandatory: Please provide 1-5 keywords
% Author macros::end %%%%%%%%%%%%%%%%%%%%%%%%%%%%%%%%%%%%%%%%%%%%%%%%%
%
%Editor-only macros:: begin (do not touch as author)%%%%%%%%%%%%%%%%%%%%%%%%%%%%%%%%%%

\EventEditors{S\'andor Fekete and Anna Lubiw}
\EventNoEds{2}
\EventLongTitle{32nd International Symposium on Computational Geometry
	(SoCG 2016)}
\EventShortTitle{SoCG 2017}
\EventAcronym{SoCG}
\EventYear{2016}
\EventDate{Jul7 4-6, 2016}
\EventLocation{Brisbane, Australia}
\EventLogo{}
\SeriesVolume{52}
\ArticleNo{66}    %Proceedings number goes here!

\begin{document}

\maketitle

\begin{abstract}

 \end{abstract}

\section{Introduction}

\section{Algorithm}
We begin by receiving an uploaded target image, $I^{t}$, and a query term used to retrieve a second source image,$I^{s}$ . Then, we triangulate each image into a set of colored triangles $T^t = \{\Delta^t_1, \cdots, \Delta^t_n\}$ and $T^s = \{\Delta^s_1, \cdots, \Delta^s_m\}$ . Next we use $k$-means to cluster the colors in $I^t$ and $I^s$ to get the top $k$ colors of each \cite[k-means] by extracting the color centroids from each clustering. Denote these sets $\{C^t_1,\cdots, C^t_k \}$ and $\{C^s_1,\cdots, C^s_k \}$. We then pair up these colors by implementing weighted bipartite matching on the graph with the edge weight of edge $[C^t_i, C^s,j] = |H(C^t_i) - H(C^s_j)|$ where $H(C)$ is the hue of color $C$.

Now, for both triangulations $T^t$ and $T^s$, we recolor the triangles using the color redistrubtion algorithm and get $T^s_1$ and $T^s_1$ to ensure the total areas of each color is the same for all $k$ colors across all triangles. Given these color balanced triangulations $T^t_1$ and $T^s_1$ we now recolor each triangle in $T^s_1$ according to the matching described above and get  $T^s_2$. Finally, we use scissors congruence to recreate $P_{target}$ from $P_{base}$. 

\subsection{Color Redistribution Algorithm}
Given a colored triangulation $T$ and color centroids $C=\{C_1,\cdots, C_k \}$, first
for each triangle $\Delta$ in $T$ we compute the average hue of the pixels in its interior, $\bar{H}(\Delta)$ and then color it with $C_j = \arg \min_C |\bar{H}(\Delta)-H(C_i)|$, the closest of the $k$ colors centroids. Then we  balance the total area of each color, by creating $k$ bins $B_j$ with an area threshold of $\sigma=\frac{1}{k}\sum_{\Delta \in T}\text{area}(\Delta)$ and an area $$\text{area$(B_j)$} = \sum_{\Delta \in T} (\text{area}(\Delta) | \text{$\Delta$ is colored $C_j$}).$$ Now, for a given bin $B_j$, if $\text{area$(B_j)$} > \delta$, then we consider the smallest triangle $\Delta_{min} \in B_j$. If $\text{area$(B_j)$} - \text{area$(\delta_{min})$} \geq \delta$ then we recolor $\Delta_{min}$ to the bin with the closest hue $B_\ell$ with $\text{area$(B_\ell)$} < \delta$ . If $\text{area$(B_j)$} - \text{area$(\delta_{min})$} < \delta$, that is $\Delta_{min}$ cannot be removed with diminishing the area of $B_j$ below $\delta$, we instead consider the largest triangle $\Delta_{max}\in B_j$. We divide $t_max$ into two smaller triangles by drawing a ray from a vertex such that one of the new triangles $\Delta_{max}'$ has area $\text{area$(B_j)$} - \delta $ and recolor $\Delta_{max}'$ as described above. This process is repeated until there is are no bins to place new triangles in.
\section{Implementation}


%%
%% Bibliography
%%

%% Either use bibtex (recommended), but commented out in this sample

%\bibliography{dummybib}

%% .. or use bibitems explicitely

\nocite{Simpson}

\begin{thebibliography}{50}
\bibitem{wallace}William Wallace and John Lowry. ‘Question 269’. \textit{New Series of the Mathematical Repository 3} (1814). Ed. by Thomas Leybourn, pp. 44–46.
\end{thebibliography}


\end{document}
